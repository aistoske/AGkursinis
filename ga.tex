\documentclass[a4paper]{article}

\usepackage[utf8]{inputenc}
\usepackage[L7x]{fontenc}
\usepackage[lithuanian]{babel}
\usepackage{lmodern}

\newcommand{\mik}{MiK\TeX}

\title{\textbf{Kursinio darbo \\ "Grynųjų pinigų prognozavimas" planas}}

\date{2011 m. rugsėjo 21 d.}

\begin{document}
\maketitle
\maketitle


\begin{flushleft}
• I punktas. Susipažinti su teorine medžiaga. Išsiaiškinti, kas yra grynieji pinigai, kokia grynųjų pinigų problematika, kam reikalingas grynųjų pinigų prognozavimas, bei nuo ko jis priklauso. 
\end{flushleft}

\medskip


\begin{flushleft}
• II punktas. Surasti naujausius duomenis: Lietuvos, Latvijos, Estijos grynųjų pinigų kiekį, BVP, infliacijos, bei palūkanų normos rodiklius. Juos susisteminti ir išanalizuoti. Tam naudosime R programą. (Tai planuojame atlikti iki spalio 3d.)
\end{flushleft}

\medskip

\begin{flushleft}
• III punktas. Parinkti tinkamiausią modelį. jį patikrinti ir išanalizuoti. Pateikti gautų rezultatų interpretacijas. Jei pavyks, patikrinti pinigų kiekio ir kitų ūkio indikatorių sąsajas.
\end{flushleft}

\medskip

\begin{flushleft}
• IV punktas. Atlikti Baltijos šalių pinigų kiekio prognozę metams į priekį, ją pakomentuoti ir pateikti išvadas. (Tai planuojame atlikti iki lapkričio 4d.)
\end{flushleft}

\medskip
\medskip
\medskip
\medskip
\medskip
\medskip
\medskip
\begin{flushright}
\textbf{Greta Petrulaitytė, Aistė Dapkūnaitė, EKO1}
\end{flushright}



\end{document}